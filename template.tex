% This is the Reed College LaTeX thesis template. Most of the work
% for the document class was done by Sam Noble (SN), as well as this
% template. Later comments etc. by Ben Salzberg (BTS). Additional
% restructuring and APA support by Jess Youngberg (JY).
% Your comments and suggestions are more than welcome; please email
% them to cus@reed.edu
%
% See https://www.reed.edu/cis/help/LaTeX/index.html for help. There are a
% great bunch of help pages there, with notes on
% getting started, bibtex, etc. Go there and read it if you're not
% already familiar with LaTeX.
%
% Any line that starts with a percent symbol is a comment.
% They won't show up in the document, and are useful for notes
% to yourself and explaining commands.
% Commenting also removes a line from the document;
% very handy for troubleshooting problems. -BTS

% As far as I know, this follows the requirements laid out in
% the 2002-2003 Senior Handbook. Ask a librarian to check the
% document before binding. -SN

%%
%% Preamble
%%
% \documentclass{<something>} must begin each LaTeX document
\documentclass{ufdissertation} %UF's 2019 Template --ANF
% Packages are extensions to the basic LaTeX functions. Whatever you
% want to typeset, there is probably a package out there for it.
% Chemistry (chemtex), screenplays, you name it.
% Check out CTAN to see: https://www.ctan.org/
%%
\usepackage{siunitx}
% \usepackage{graphicx,latexsym}
% \usepackage{booktabs}
% \usepackage{chemarr} %% Useful for one reaction arrow, useless if you're not a chem major
% \usepackage[hyphens]{url}
% % Added by CII
% \usepackage{lmodern}
% \floatplacement{figure}{H}
% % Thanks, @Xyv
% \usepackage{calc}
% % End of CII addition
% \usepackage{rotating}

% Next line commented out by CII
%%% \usepackage{natbib}
% Comment out the natbib line above and uncomment the following two lines to use the new
% biblatex-chicago style, for Chicago A. Also make some changes at the end where the
% bibliography is included.
%\usepackage{biblatex-chicago}
%\bibliography{thesis}


% % Added by CII (Thanks, Hadley!)
% % Use ref for internal links
% \renewcommand{\hyperref}[2][???]{\autoref{#1}}
% \def\chapterautorefname{Chapter}
% \def\sectionautorefname{Section}
% \def\subsectionautorefname{Subsection}
% % End of CII addition

% \usepackage{times} % other fonts are available like times, bookman, charter, palatino

% Syntax highlighting #22
$if(highlighting-macros)$
  $highlighting-macros$
$endif$

% To pass between YAML and LaTeX the dollar signs are added by CII
\title{$title$}
\author{$author$}
% The month and year that you submit your FINAL draft TO THE LIBRARY (May or December)
\degreeyear{$year$}
\degreemonth{$month$}
% \division{$division$}
\chair{$chair$}
%If you have two advisors for some reason, you can use the following
% Uncommented out by CII
\cochair{$cochair$}
% End of CII addition
% \institution{$institution$}
\degreetype{$degree$}
\major{$major$}
\thesistype{$thesistype$}
%%% Remember to use the correct department!
% \department{$department$}
% if you're writing a thesis in an interdisciplinary major,
% uncomment the line below and change the text as appropriate.
% check the Senior Handbook if unsure.
%\thedivisionof{The Established Interdisciplinary Committee for}
% if you want the approval page to say "Approved for the Committee",
% uncomment the next line
%\approvedforthe{Committee}

% Added by CII
%%% Copied from knitr
%% maxwidth is the original width if it's less than linewidth
%% otherwise use linewidth (to make sure the graphics do not exceed the margin)
% \makeatletter
% \def\maxwidth{ %
%   \ifdim\Gin@nat@width>\linewidth
%     \linewidth
%   \else
%     \Gin@nat@width
%   \fi
% }
% \makeatother

% % From {rticles}
% $if(csl-refs)$
% \newlength{\csllabelwidth}
% \setlength{\csllabelwidth}{3em}
% \newlength{\cslhangindent}
% \setlength{\cslhangindent}{1.5em}
% % for Pandoc 2.8 to 2.10.1
% \newenvironment{cslreferences}%
%   {$if(csl-hanging-indent)$\setlength{\parindent}{0pt}%
%   \everypar{\setlength{\hangindent}{\cslhangindent}}\ignorespaces$endif$}%
%   {\par}
% % For Pandoc 2.11+
% % As noted by @mirh [2] is needed instead of [3] for 2.12
% \newenvironment{CSLReferences}[2] % #1 hanging-ident, #2 entry spacing
%  {% don't indent paragraphs
%   \setlength{\parindent}{0pt}
%   % turn on hanging indent if param 1 is 1
%   \ifodd #1 \everypar{\setlength{\hangindent}{\cslhangindent}}\ignorespaces\fi
%   % set entry spacing
%   \ifnum #2 > 0
%   \setlength{\parskip}{#2\baselineskip}
%   \fi
%  }%
%  {}
% \usepackage{calc} % for calculating minipage widths
% \newcommand{\CSLBlock}[1]{#1\hfill\break}
% \newcommand{\CSLLeftMargin}[1]{\parbox[t]{\csllabelwidth}{#1}}
% \newcommand{\CSLRightInline}[1]{\parbox[t]{\linewidth - \csllabelwidth}{#1}}
% \newcommand{\CSLIndent}[1]{\hspace{\cslhangindent}#1}
% $endif$

% \renewcommand{\contentsname}{Table of Contents}
% End of CII addition

% \setlength{\parskip}{0pt}

% % Added by CII
% $if(space_between_paragraphs)$
%   %\setlength{\parskip}{\baselineskip}
%   \usepackage[parfill]{parskip}
% $endif$

% \providecommand{\tightlist}{%
%   \setlength{\itemsep}{0pt}\setlength{\parskip}{0pt}}

\begin{document}

%%%%%%%%%%%%%%%%%%%%%%%%%%%%%%%%%
% TITLE PAGE                    % 
%%%%%%%%%%%%%%%%%%%%%%%%%%%%%%%%%

\maketitle                       % Titlepage comes first
    \newpage
      \vspace*{\fill}
        \begin{center}
            \textcopyright{} \@degreeyear{} \@author
        \end{center}
      \vspace*{\fill}
    \newpage

\setcounter{secnumdepth}{-1}     % We don't want chapter numbers until later, 
                                 % So let's kill off the table of contents depth detector until we want to start counting.


%%%%%%%%%%%%%%%%%%%%%%%%%%%%%%%%%
% COPYRIGHT PAGE                %
%%%%%%%%%%%%%%%%%%%%%%%%%%%%%%%%%



%%%%%%%%%%%%%%%%%%%%%%%%%%%%%%%%%
% DEDICATION                    %
%%%%%%%%%%%%%%%%%%%%%%%%%%%%%%%%%

  \begin{dedication}
    \dedication{$dedication$}
  \end{dedication}


%%%%%%%%%%%%%%%%%%%%%%%%%%%%%%%%%
% ACKNOWLEDGMENTS               %
%%%%%%%%%%%%%%%%%%%%%%%%%%%%%%%%%


\begin{acknowledgments}
    \acknowledgments{$acknowledgments$}
\end{acknowledgments}


%%%%%%%%%%%%%%%%%%%%%%%%%%%%%%%%%
% TABLE OF CONTENTS             %
%%%%%%%%%%%%%%%%%%%%%%%%%%%%%%%%%
\realSingleSpace\tableofcontents  % Table of Contents comes fourth.
    
%%%%%%%%%%%%%%%%%%%%%%%%%%%%%%%%%
% LIST OF TABLES                %
%%%%%%%%%%%%%%%%%%%%%%%%%%%%%%%%%

\realSingleSpace\listoftables    % List of tables comes next, if you have one.
\addcontentsline{toc}{chapter}{LIST OF TABLES}
        

%%%%%%%%%%%%%%%%%%%%%%%%%%%%%%%%%
% LIST OF FIGURES               %
%%%%%%%%%%%%%%%%%%%%%%%%%%%%%%%%%
\realSingleSpace\listoffigures   % List of figures comes next, if you have one.
\addcontentsline{toc}{chapter}{LIST OF FIGURES}

%%%%%%%%%%%%%%%%%%%%%%%%%%%%%%%%%
% LIST OF ABBREVIATIONS         %
%%%%%%%%%%%%%%%%%%%%%%%%%%%%%%%%%

%%%%%%%%%%%%%%%%%%%%%%%%%%%%%%%%%
% ACADEMIC ABSTRACT             %
%%%%%%%%%%%%%%%%%%%%%%%%%%%%%%%%%

\abstract{$abstract$}

%%%%%%%%%%%%%%%%%%%%%%%%%%%%%%%%%
% CHAPTERS                      %
%%%%%%%%%%%%%%%%%%%%%%%%%%%%%%%%%
\titlecontents{chapter}[2em]
{\addvspace{8pt}}
{\hspace*{-2em}\hyper@linkstart{link}{\Hy@tocdestname}\hspace*{2em}{\contentslabel{2em}}\hyper@linkend}%
{\hyper@linkstart{link}{\Hy@tocdestname}{\titlerule*[5pt]{.}\thecontentspage}\hyper@linkend \\*}

$body$


%%%%%%%%%%%%%%%%%%%%%%%%%%%%%%%%%
% APPENDICES                    %
%%%%%%%%%%%%%%%%%%%%%%%%%%%%%%%%%
\titlecontents{chapter}[0em]
{\addvspace{8pt}}
{\hspace*{-2em}\hyper@linkstart{link}{\Hy@tocdestname}\hspace*{2em}{APPENDIX:  \expandafter\contentslabel{2em}}\hyper@linkend%
{\hyper@linkstart{link}{\Hy@tocdestname}{\titlerule*[5pt]{.}\thecontentspage}\hyper@linkend \\*}
\renewcommand{\thechapter}{}     % Remove the number in front of the appendix chapter name.
\titlecontents{chapter}[0em]
{\addvspace{8pt}}
{\hspace*{-2em}\hyper@linkstart{link}{\Hy@tocdestname}\hspace*{2em}{APPENDIX: }\hyper@linkend%
{\hyper@linkstart{link}{\Hy@tocdestname}{\titlerule*[5pt]{.}\thecontentspage}\hyper@linkend \\*}
\renewcommand{\thechapter}{}     % Remove the number in front of the appendix chapter name.
\titleformat{\chapter}[hang]
{\uppercase}
{}
{0pt}
{\centering\realSingleSpace Appendix\\[-5pt]}
[\raggedright\doublespacing]

\setcounter{secnumdepth}{4}
\newcounter{appendixVal}         % Use an entirely different counter for Appendicies to avoid
                                 %  hyperref misdirections to other already existing chapters.
\setcounter{appendixVal}{0}      %  Start counter at 0,
                                 %  Since the counter is stepped before the label is made.
\let\appendixChapter\chapter     % Save the normal \chapter command so we can temporarily hijack it.
\renewcommand{\chapter}[1]{\stepcounter{appendixVal}\appendixChapter{#1}}% Append a stepcounter to the Appendix
\renewcommand{\thechapter}{\Alph{appendixVal}}%
\input{$appendix$}               % Imports Appendix from Rmarkdown
\let\chapter\appendixChapter%

\titleformat{\chapter}[hang]%
{\uppercase}
{}
{0pt}
% {\centering\realSingleSpace\ifdocBody CHAPTER \thechapter \\[-5pt] \fi}
[\raggedright\doublespacing]



%%%%%%%%%%%%%%%%%%%%%%%%%%%%%%%%%
% LIST OF REFERENCES            %
%%%%%%%%%%%%%%%%%%%%%%%%%%%%%%%%%

% provided by file 98-references.Rmd

%%%%%%%%%%%%%%%%%%%%%%%%%%%%%%%%%
% BIOGRAPHICAL SKETCH
%%%%%%%%%%%%%%%%%%%%%%%%%%%%%%%%%

\chapter{BIOGRAPHICAL SKETCH}
\label{biography}
\vspace*{-0.5\baselineskip}      % Offset the doublespacing between title and paragraph.
\input{\$biography$}
        
\setlength\parindent{1cm}


\end{document}
