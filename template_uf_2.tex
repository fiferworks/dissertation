% This is an adaptation of the the Reed College LaTeX thesis template.
% Most of the work for the document class was done by Sam Noble (SN),
% as well as this  template. Later comments etc. by Ben Salzberg (BTS).
% Additional restructuring and APA support by Jess Youngberg (JY).
% Your comments and suggestions are more than welcome; please email
% them to cus@reed.edu
%
% See https://www.reed.edu/cis/help/LaTeX/index.html for help. There are a
% great bunch of help pages there, with notes on
% getting started, bibtex, etc. Go there and read it if you're not
% already familiar with LaTeX.
%
% Any line that starts with a percent symbol is a comment.
% They won't show up in the document, and are useful for notes
% to yourself and explaining commands.
% Commenting also removes a line from the document;
% very handy for troubleshooting problems. -BTS
%
%
% This template was originally adapted by Kristen Sauby (KS) in 2017
% to meet the requirements outlined in the University of Florida
% Guide for Preparing Theses and Dissertations
% http://graduateschool.ufl.edu/media/graduate-school/pdf-files/Guide-for-ETDs.pdf
% http://graduateschool.ufl.edu/about-us/offices/editorial/thesis-and-dissertation/
% https://helpdesk.ufl.edu/application-support-center/etd-technical-support/ms-word-and-latex-templates/
%
% Austin N Fife (AF) made some adaptations to the template and .cls files in 2021 to reduce
% dependencies on external files, as well as integrating many of the edits from ismayc's depository
%%
%% Preamble and comments added by AF
%%
% \documentclass{<something>} must begin each LaTeX document
\documentclass[12pt,final,CPage]{ufdissertation}

% moved packages from 'packages.tex' to the template file -AF
% here you define all the packages you wish to use in your paper, the ones shown are not all necessary,
% but all have purpose and can be very useful, so leave these as default and add packages as necassary
\usepackage{graphicx}
\usepackage{microtype}
\usepackage{algpseudocode}
\usepackage{lscape}
\usepackage{amsfonts}
\usepackage{textcomp}
\usepackage[nottoc]{tocbibind}
\usepackage{booktabs}
\usepackage{threeparttable}
\usepackage{subfigure}
\usepackage{placeins}
\usepackage{rotating}
\usepackage{calc}
\usepackage{setspace}
%\usepackage{ufenumerate}
\usepackage{latexsym}
\usepackage{epsf}
\usepackage{epsfig}
\usepackage{euscript}
\usepackage[format=hang,justification=raggedright,singlelinecheck=0,labelsep=period]{caption}
\usepackage{siunitx}
\usepackage{textgreek}
\usepackage{pdfpages}
\usepackage[defaultlines=4,all]{nowidow}
\raggedbottom
% \usepackage{letltxmacro}
% \LetLtxMacro{\oldhypertarget}{\hypertarget}
% \renewcommand{\hypertarget}[2]{\leavevmode\oldhypertarget{#1}{#2}}
% % %DO NOT PLACE ANY PACKAGES AFTER THE HYPERREF SET UP

%from ismayc/thesisdown, allows for changes made to newer release of pandoc
% From {rticles}
$if(csl-refs)$
\newlength{\csllabelwidth}
\setlength{\csllabelwidth}{3em}
\newlength{\cslhangindent}
\setlength{\cslhangindent}{1.5em}
% for Pandoc 2.8 to 2.10.1
\newenvironment{cslreferences}%
{$if(csl-hanging-indent)$\setlength{\parindent}{0pt}%
	\everypar{\setlength{\hangindent}{\cslhangindent}}\ignorespaces$endif$}%
{\par}
% For Pandoc 2.11+
\newenvironment{CSLReferences}[2] % #1 hanging-ident, #2 entry spacing
{% don't indent paragraphs
	\setlength{\parindent}{0pt}
	% turn on hanging indent if param 1 is 1
	\ifodd #1 \everypar{\setlength{\hangindent}{\cslhangindent}}\ignorespaces\fi
	% set entry spacing
	\ifnum #2 > 0
	\setlength{\parskip}{#2\baselineskip}
	\fi
}%
{}
\usepackage{calc} % for calculating minipage widths
\newcommand{\CSLBlock}[1]{#1\hfill\break}
\newcommand{\CSLLeftMargin}[1]{\parbox[t]{\csllabelwidth}{#1}}
\newcommand{\CSLRightInline}[1]{\parbox[t]{\linewidth - \csllabelwidth}{#1}}
\newcommand{\CSLIndent}[1]{\hspace{\cslhangindent}#1}
$endif$

$if(highlighting-macros)$
$highlighting-macros$
$endif$

\def\UrlFont{\rmfamily}

\renewcommand{\topfraction}{0.85}
\renewcommand{\textfraction}{0.1}
\renewcommand{\floatpagefraction}{0.75}

%this grabs the information provided in your yaml header and formats it using the .cls file
\author{$author$}
\thesisType{Dissertation}
\degreeType{$degree$}
\degreeMonth{$month$}
\degreeYear{$date$}
\major{$department$}
\chair{$advisor$, $altadvisor$}
\title{$title$}

%\include{usersetcommands}

\begin{document}

	%\bibliographystyle{IEEEtran} %I think we should let the rmarkdown .csl define the formatting instead -AF

%this creates and formats your title page using the data supplied from your yaml header
	% \maketitle

	\setCopyrightFile

%this creates and formats your dedication page using the data supplied from your yaml header
	\setDedicationFile{$dedication$}

%this creates and formats your acknowledgements using the data supplied from your yaml header
	\setAcknowledgementsFile{$acknowledgements$}

%The list shown below gives a brief description of the major mathematical symbols defined in this work. For each
%symbol, the page number corresponds to the place where the symbol is first used.} %
\microtypesetup{protrusion=false}
\tableofcontents

\listoftables % comment out if you don't have tables
\microtypesetup{protrusion=false}

\listoffigures % comment out if you don't have figures
  %\setcounter{lofdepth}{2}
\microtypesetup{protrusion=true}

% Produced list of abbreviations or symbols
% These commands don't work now, the TOC used to be preserved as a separate tex file
% hopefully someone who needs these can find a way to fix them
% \printindex[keylist]{KEY TO ABBREVIATIONS}{KEY TO ABBREVIATIONS}{}
% \printindex[mathlist]{KEY TO SYMBOLS}{KEY TO SYMBOLS}{%
% }% Produced list of abbreviations or symbols %
%\printindex[keylist]{KEY TO ABBREVIATIONS}{KEY TO ABBREVIATIONS}{}
%\printindex[mathlist]{KEY TO SYMBOLS}{KEY TO SYMBOLS}{%
%\phantomsection

%this takes the 'abstract:' field from your yaml header and formats it
  \setAbstractFile{$abstract$}

%this adds a header which says 'CHAPTER' to the toc
  \addtocontents{toc}{\protect\addvspace{10pt}\noindent{CHAPTER}\protect\hfill\par}{\pdfbookmark[0]{TABLE OF CONTENTS}{tableofcontents}
  $body$

%formats and injects the bibliography here
\bibliography {bib/sample,bib/example}

%this takes the 'preface:' field from your yaml header and uses it as the biography
%removes the dependency on the .tex file
  \setBiographicalFile{$preface$}

\end{document}

%------------------------------------------------------------------------------%
