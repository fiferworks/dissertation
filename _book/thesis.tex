% This is an adaptation of the the Reed College LaTeX thesis template.
% Most of the work for the document class was done by Sam Noble (SN),
% as well as this  template. Later comments etc. by Ben Salzberg (BTS).
% Additional restructuring and APA support by Jess Youngberg (JY).
% Your comments and suggestions are more than welcome; please email
% them to cus@reed.edu
%
% See https://www.reed.edu/cis/help/LaTeX/index.html for help. There are a
% great bunch of help pages there, with notes on
% getting started, bibtex, etc. Go there and read it if you're not
% already familiar with LaTeX.
%
% Any line that starts with a percent symbol is a comment.
% They won't show up in the document, and are useful for notes
% to yourself and explaining commands.
% Commenting also removes a line from the document;
% very handy for troubleshooting problems. -BTS
%
%
% This template was originally adapted by Kristen Sauby (KS) in 2017
% to meet the requirements outlined in the University of Florida
% Guide for Preparing Theses and Dissertations
% http://graduateschool.ufl.edu/media/graduate-school/pdf-files/Guide-for-ETDs.pdf
% http://graduateschool.ufl.edu/about-us/offices/editorial/thesis-and-dissertation/
% https://helpdesk.ufl.edu/application-support-center/etd-technical-support/ms-word-and-latex-templates/
%
% Austin N Fife (AF) made some adaptations to the template and .cls files in 2021 to reduce
% dependencies on external files, as well as integrating many of the edits from ismayc's depository
%%
%% Preamble and comments added by AF
%%
% \documentclass{<something>} must begin each LaTeX document
\documentclass[12pt,final,CPage]{ufthesis}

% moved packages from 'packages.tex' to the template file -AF
% here you define all the packages you wish to use in your paper, the ones shown are not all necessary,
% but all have purpose and can be very useful, so leave these as default and add packages as necassary
\usepackage{graphicx}
\usepackage{microtype}
%\usepackage[dvipdfmx]{graphicx}
\usepackage{amsmath}
\usepackage{amsthm}
\usepackage{algpseudocode}
\usepackage{tabularx}
\usepackage{url}
\usepackage[letterpaper,hmargin=1in,vmargin=1in]{geometry}
\usepackage{lscape}
%\usepackage{hanging}
\usepackage{float}
\usepackage{longtable}
\usepackage[table]{xcolor}
\usepackage{amsfonts}
\usepackage{amssymb}
\usepackage{textcomp}
%\usepackage[cmbright]{sfmath} % Comment this line to use Times New Roman Math Typeface\\
\usepackage[nottoc]{tocbibind}
\usepackage{booktabs}
\usepackage{threeparttable}
\usepackage{subfigure}
\usepackage{flafter}
\usepackage{placeins}
\usepackage{rotating}
\usepackage{calc}
\usepackage{setspace}
%\usepackage{ufenumerate}
\usepackage{latexsym}
\usepackage{epsf}
\usepackage{epsfig}
\usepackage{euscript}
\usepackage[format=hang,justification=raggedright,singlelinecheck=0,labelsep=period]{caption}
\usepackage[numbers,sort&compress]{natbib} %Use this set-up for numbered reference lists
%\usepackage[authoryear]{natbib} %Use this set-up if you want an un-numbered reference list
%\usepackage{hypernat}
\usepackage{siunitx}
%\usepackage[utf8]{inputenc}
\usepackage{textgreek}
\usepackage{pdfpages}
\RequirePackage[linktoc=all]{hyperref}% Use this to provide intra-pdf hyperlinking and better toc
\hypersetup{%               %           Setup the coloring of the links.
%                           %           Currently the only necessary one is "colorlinks=true" and "linkcolor=blue".
    colorlinks   = true,    %           Colours links instead of ugly boxes
    urlcolor     = blue,    %           Colour for external hyperlinks
    linkcolor    = blue,    %           Colour of internal links
    citecolor    = blue     %           Colour of citations, could be ``red''
    }
% % %DO NOT PLACE ANY PACKAGES AFTER THE HYPERREF SET UP

%from ismayc/thesisdown, allows for changes made to newer release of pandoc
% From {rticles}


\def\UrlFont{\rmfamily}

\renewcommand{\topfraction}{0.85}
\renewcommand{\textfraction}{0.1}
\renewcommand{\floatpagefraction}{0.75}

%this grabs the information provided in your yaml header and formats it using the .cls file
\SetFullName{Austin Nathaniel Fife}
\SetThesisType{Dissertation}
\SetDegreeType{Doctor of Philosophy}
\SetGradMonth{}
\SetGradYear{2021}
\SetDepartment{Entomology and Nematology}
\SetChair{Xavier Martini}
\SetCochair{Mathews Paret}
\SetTitle{Mite-Virus-Plant Complexes of Importance for Florida Agriculture: Early Detection, Chemical Ecology and Biocontrol of \emph{Phyllocoptes fructiphilus} and \emph{Brevipalpus californicus}}

%\include{usersetcommands}
\begin{document}

	%\bibliographystyle{IEEEtran} %I think we should let the rmarkdown .csl define the formatting instead -AF

%this creates and formats your title page using the data supplied from your yaml header
	\maketitle

	\makecopyright

%this creates and formats your dedication page using the data supplied from your yaml header
	\dedication{`For Liz, Violet, Juniper and Fifes to come'}

%this creates and formats your acknowledgements using the data supplied from your yaml header
	\acknowledge{`I would like to give special thanks for the Tallahassee Museum and their patience, cooperation, and support with collecting plant samples. I am grateful for the USDA-APHIS PPQ Beltsville laboratory for their support in the identification and confirmation OFV isolates, as well as \emph{Brevipalpus} mite identification at the USDA-ARS. I also want to thank Drs. Sam Bolton, FDACS and Aline Tassi, Univ. of Sao Paulo, Brazil for checking the mites we have sent for species validation. Furthermore, I am grateful for Dr.~Marc S. Frank's identification of the Liriopogons collected. I am especially indebted to the late Dr.~Gary Bauchan for his contributions to this study and the field of acarology, he will be greatly missed. This research was partly funded by the USDA National Institute of Food and Agriculture, Hatch project FLA-NFC-005607. Mention of trade names or commercial products in this publication is solely for the purpose of providing specific information and does not imply recommendation or endorsement by the USDA; USDA is an equal opportunity provider and employer.'}

%The list shown below gives a brief description of the major mathematical symbols defined in this work. For each
%symbol, the page number corresponds to the place where the symbol is first used.} %
\microtypesetup{protrusion=false}
\tableofcontents

\listoftables % comment out if you don't have tables
\microtypesetup{protrusion=false}

\listoffigures % comment out if you don't have figures
  %\setcounter{lofdepth}{2}
\microtypesetup{protrusion=true}

% Produced list of abbreviations or symbols
% These commands don't work now, the TOC used to be preserved as a separate tex file
% hopefully someone who needs these can find a way to fix them
% \printindex[keylist]{KEY TO ABBREVIATIONS}{KEY TO ABBREVIATIONS}{}
% \printindex[mathlist]{KEY TO SYMBOLS}{KEY TO SYMBOLS}{%
% }% Produced list of abbreviations or symbols %
%\printindex[keylist]{KEY TO ABBREVIATIONS}{KEY TO ABBREVIATIONS}{}
%\printindex[mathlist]{KEY TO SYMBOLS}{KEY TO SYMBOLS}{%
  \phantomsection

%this takes the 'abstract:' field from your yaml header and formats it
  \abstract{'Rose Rosette Virus (genus Emaraviridae) is the most devastating disease of roses. Rose Rosette Virus (RRV) creates witches brooms, rosetting, deforms flowers, increases prickle density, elongates shoots, reddens of plant tissues, causes dieback and ultimately plant death. RRV is spread by a microscopic eriophyid mite known as Phyllocoptes fructiphilus Keifer (Trombidiformes: Eriophyidae). Few management options are available: Current mite control is achieved by removing infected roses and frequent pesticide applications. Growers are interested in alternative and less expensive management options to combat P. fructiphilus and RRV. Aggressive pruning was tested for its ability to reduce populations of P. fructiphilus. Mites from the family Phytoseiidae are being investigated as biocontrol agents for the management of P. fructiphilus. A survey of mites on roses was conducted in northern Florida to search for P. fructiphilus, RRD and/or predatory mites. Preliminary data suggest that the phytoseiid mite Amblyseius swirskii Athias-Henriot (Mesostigmata: Phytoseiidae) orients itself towards volatiles of RRV-infected roses. This attraction may have synergistic effects for *P. fructiphilus control. A. swirskii was tested in olfactometer choice tests to identify specific volatile compounds which may cause this behavior. Low levels of Methyl Salicylate found in RRV-infected roses suggest that the virus interferes with the rose's ability to defend itself against the pathogen. A avoid this negative feedback loop is to induce Systemic Acquired Resistance (SAR) before pathogen introduction, a procedure which could increase the rose's resistance to RRV. Considering this knowledge, we collaborated with the University of Georgia in to test how SAR-induction might protect roses from P. fructiphilus and/or RRD. Acibenzolar-S-methyl (ASM) is a benzothiadiazole, a SAR-inducing chemical which works like Salicylic Acid to induce plant defenses against viruses, bacteria, and fungal infection. ASM application also has shown chitinase activity in roses, and some studies have shown that the hypersensitive response and SAR interfere with the ability of eriophyoid mites to feed or grow on induced plants. A remaining concern is that SAR-induction may harm predatory mites via direct and indirect effects of SAR-induction. We conducted several field studies from 2018-2021 in order to test the integration of predatory mites with SAR. This research will contribute a biocontrol option for the management of P. fructiphilus in southern Georgia and northern Florida. We describe the first detection of orchid fleck virus (OFV) infecting three unreported hosts: Liriope muscari, cv. `Gigantea' (Decaisne) Bailey, Ophiopogon intermedius Don and Aspidistra elatior Blume (Asparagaceae: Nolinoidaea) in Leon and Alachua Counties, FL. The orchid-infecting subgroup (Orc) of OFV infects over 50 plant species belonging to the plant families Orchidaceae, Asparagaceae (Nolinoidaea), and causes citrus leprosis disease in Citrus (Rutaceae). The only known vectors of OFV-Orc are the flat mites, Brevipalpus californicus (Banks) sensu lato (Trombidiformes: Tenuipalpidae). Florida has various plants in the landscape which Brevipalpus spp. feed on, which are susceptible to infection by OFV-Orc. Chlorotic ringspots and flecking were seen affecting Liriopogons (Liriope and Ophiopogon spp.) in Leon County, FL. Nearby A. elatior also appeared chlorotic. Local diagnostics returned negative for common plant pathogens, therefore new samples were sent to the Florida Department of Agriculture and Consumer Services (FDACS) and USDA-ARS for identification. Two orchid-infecting strains of OFV were detected via combinations of conventional RT-PCR, RT-qPCR, Sanger sequencing and High Throughput Sequencing (HTS). Amplicons shared 98\% nucleotide identity with OFV-Orc1 and OFV-Orc2 RNA2 genome sequences available in NCBI GenBank. Coinfections were detected in each county, but single strains of OFV-Orc were detected in L. muscari (Alachua, OFV-Orc2) and A. elatior (Leon, OFV-Orc1). Three potential mite vectors were identified via cryo-scanning electron microscopy (Cryo-SEM): Brevipalpus californicus (Banks) sensu lato, B. obovatus Donnadieu, and B. confusus Baker. In conclusion, OFV orchid strains are present in northern Florida, representing a risk for susceptible plants in the southeastern US.'}

%this adds a header which says 'CHAPTER' to the toc
  \addtocontents{toc}{\protect\addvspace{10pt}\noindent{CHAPTER}\protect\hfill\par}{\pdfbookmark[0]{TABLE OF CONTENTS}{tableofcontents}
  \hypertarget{if-you-have-more-two-advisors-un-silence-line-9}{%
  \chapter{If you have more two advisors, un-silence line 9:}\label{if-you-have-more-two-advisors-un-silence-line-9}}

  Placeholder

  \hypertarget{review}{%
  \chapter{LITERATURE REVIEW}\label{review}}

  Placeholder

  \hypertarget{litrev-acari}{%
  \section{A small introduction to some herbivorous acari}\label{litrev-acari}}

  \hypertarget{litrev-erios}{%
  \subsection{Co-evolved plant specialists: the eriophyoidea}\label{litrev-erios}}

  \hypertarget{litrev-pfruct}{%
  \subsection{\texorpdfstring{\emph{Phyllocoptes fructiphilus}: the vector of Rose Rosette Virus, the causal agent of Rose Rosette Disease}{Phyllocoptes fructiphilus: the vector of Rose Rosette Virus, the causal agent of Rose Rosette Disease}}\label{litrev-pfruct}}

  \hypertarget{litrev-ipm}{%
  \section{Integrated pest management: best practices for modern agriculture}\label{litrev-ipm}}

  \hypertarget{litrev-manage}{%
  \subsection{Current Management of RRD is not effective}\label{litrev-manage}}

  \hypertarget{litrev-preds}{%
  \subsection{Phytoseiids mites: good options for biological control of mites?}\label{litrev-preds}}

  \hypertarget{litrev-plantdef}{%
  \section{Induced plant defenses for biological control}\label{litrev-plantdef}}

  \hypertarget{litrev-sar}{%
  \subsection{Can systemic acquired resistance be used to reduce mite herbivory?}\label{litrev-sar}}

  \hypertarget{litrev-brevi}{%
  \section{\texorpdfstring{A second mite-plant-pathogen system: \emph{Brevipalpus californicus} and Orchid fleck virus}{A second mite-plant-pathogen system: Brevipalpus californicus and Orchid fleck virus}}\label{litrev-brevi}}

  \hypertarget{survey-pheno}{%
  \chapter{\texorpdfstring{SURVEY AND PHENOLOGY OF NATURAL POPULATIONS OF THE INVASIVE MITE \emph{PHYLLOCOPTES FRUCTIPHILUS} IN NORTHERN FLORIDA}{SURVEY AND PHENOLOGY OF NATURAL POPULATIONS OF THE INVASIVE MITE PHYLLOCOPTES FRUCTIPHILUS IN NORTHERN FLORIDA}}\label{survey-pheno}}

  Placeholder

  \hypertarget{intro-survey-pheno}{%
  \section{Introduction}\label{intro-survey-pheno}}

  \hypertarget{intro-survey}{%
  \section{\texorpdfstring{Surveying for \emph{P. fructiphilus}, RRD and predatory mites in northern Florida}{Surveying for P. fructiphilus, RRD and predatory mites in northern Florida}}\label{intro-survey}}

  \hypertarget{intro-pheno}{%
  \subsection{\texorpdfstring{Phenology of natural populations of \emph{P. fructiphilus} in northern Florida}{Phenology of natural populations of P. fructiphilus in northern Florida}}\label{intro-pheno}}

  \hypertarget{mm-survey-pheno}{%
  \section{Materials \& Methods}\label{mm-survey-pheno}}

  \hypertarget{mm-survey}{%
  \subsection{Mite Survey}\label{mm-survey}}

  \hypertarget{mm-pheno}{%
  \subsection{Phenology}\label{mm-pheno}}

  \hypertarget{results-survey-pheno}{%
  \section{Results}\label{results-survey-pheno}}

  \hypertarget{results-survey}{%
  \subsection{Survey}\label{results-survey}}

  \hypertarget{results-pheno}{%
  \subsection{Phenology}\label{results-pheno}}

  \hypertarget{dis-survey-pheno}{%
  \section{Discussion}\label{dis-survey-pheno}}

  \hypertarget{chemeco}{%
  \chapter{CHANGES IN HEADSPACE VOLATILES FOR RRD-INFECTED ROSES}\label{chemeco}}

  Placeholder

  \hypertarget{intro-swirskii-vocs}{%
  \section{Introduction}\label{intro-swirskii-vocs}}

  \hypertarget{intro-swirskii}{%
  \subsection{\texorpdfstring{Rose Rosette Disease, Predatory mites and Plant Defensese: Why are \emph{Amblyseius swirskii} attracted to infected roses?}{Rose Rosette Disease, Predatory mites and Plant Defensese: Why are Amblyseius swirskii attracted to infected roses?}}\label{intro-swirskii}}

  \hypertarget{mm-vocs-olfact}{%
  \section{Materials \& Methods}\label{mm-vocs-olfact}}

  \hypertarget{mm-vocs}{%
  \subsection{Collection of headspace volatiles from roses}\label{mm-vocs}}

  \hypertarget{mm-vct}{%
  \subsubsection{Volatile Collection Trap Method}\label{mm-vct}}

  \hypertarget{mm-spme}{%
  \subsubsection{Solid Phase Micro Extraction Method}\label{mm-spme}}

  \hypertarget{mm-voc-analyze}{%
  \subsubsection{Analysis of Headspace Data}\label{mm-voc-analyze}}

  \hypertarget{mm-olfact}{%
  \subsection{Two arm olfactometer assays}\label{mm-olfact}}

  \hypertarget{results-vocs-olfact}{%
  \section{Results}\label{results-vocs-olfact}}

  \hypertarget{results-vocs}{%
  \subsection{Volatile differences between infected, healthy and induced roses}\label{results-vocs}}

  \hypertarget{results-olfact}{%
  \subsection{\texorpdfstring{\emph{A. swirskii} attraction to VOCs}{A. swirskii attraction to VOCs}}\label{results-olfact}}

  \hypertarget{dis-vocs-olfact}{%
  \section{Discussion}\label{dis-vocs-olfact}}

  \hypertarget{intro-asm-ipm-pfruct}{%
  \chapter{\texorpdfstring{INTEGRATED PEST MANAGEMENT OF \emph{PHYLLOCOPTES FRUCTIPHILUS}}{INTEGRATED PEST MANAGEMENT OF PHYLLOCOPTES FRUCTIPHILUS}}\label{intro-asm-ipm-pfruct}}

  Placeholder

  \hypertarget{intro-ipm-pfruct}{%
  \section{\texorpdfstring{Introduction: \emph{Phyllocoptes fructiphiulus} - an increasingly large problem}{Introduction: Phyllocoptes fructiphiulus - an increasingly large problem}}\label{intro-ipm-pfruct}}

  \hypertarget{intro-asm-ipm}{%
  \subsection{\texorpdfstring{Integrating Pest Management: What are the effects of Systemic Acquired Resistance on \emph{Amblyseius swirskii} and \emph{P. fructiphilus}?}{Integrating Pest Management: What are the effects of Systemic Acquired Resistance on Amblyseius swirskii and P. fructiphilus?}}\label{intro-asm-ipm}}

  \hypertarget{mm-asm-ipm}{%
  \section{Materials \& Methods}\label{mm-asm-ipm}}

  \hypertarget{ipm-actigard}{%
  \subsection{\texorpdfstring{SAR-induction with ASM to reduce populations of \emph{P. fructiphilus}}{SAR-induction with ASM to reduce populations of P. fructiphilus}}\label{ipm-actigard}}

  \hypertarget{ipm-trials}{%
  \subsection{\texorpdfstring{Integrating Pest Management Methods to control \emph{Phyllocoptes fructiphilus}}{Integrating Pest Management Methods to control Phyllocoptes fructiphilus}}\label{ipm-trials}}

  \hypertarget{materials-methods}{%
  \subsection{4.2.3 Materials \& Methods}\label{materials-methods}}

  \hypertarget{results-asm-ipm}{%
  \section{Results}\label{results-asm-ipm}}

  \hypertarget{dis-asm-ipm}{%
  \section{Discussion}\label{dis-asm-ipm}}

  \hypertarget{brevipalpus-transmitted-orchid-fleck-virus-infecting-three-new-ornamental-hosts-in-florida}{%
  \chapter{\texorpdfstring{\emph{BREVIPALPUS}-TRANSMITTED ORCHID FLECK VIRUS INFECTING THREE NEW ORNAMENTAL HOSTS IN FLORIDA}{BREVIPALPUS-TRANSMITTED ORCHID FLECK VIRUS INFECTING THREE NEW ORNAMENTAL HOSTS IN FLORIDA}}\label{brevipalpus-transmitted-orchid-fleck-virus-infecting-three-new-ornamental-hosts-in-florida}}

  Placeholder

  \hypertarget{virus-detection}{%
  \section{Virus Detection}\label{virus-detection}}

  \hypertarget{a-comment-on-the-the-status-of-brevipalpus-in-florida}{%
  \section{\texorpdfstring{A comment on the the status of \emph{Brevipalpus} in Florida}{A comment on the the status of Brevipalpus in Florida}}\label{a-comment-on-the-the-status-of-brevipalpus-in-florida}}

  \hypertarget{conclusions-invasive-mites-are-an-increasingly-large-problem-for-florida}{%
  \chapter{CONCLUSIONS: INVASIVE MITES ARE AN INCREASINGLY LARGE PROBLEM FOR FLORIDA}\label{conclusions-invasive-mites-are-an-increasingly-large-problem-for-florida}}

  The principal findings of these trials are:
  \begin{enumerate}
  \def\labelenumi{\arabic{enumi}.}
  \tightlist
  \item
    \emph{Phyllocoptes fructiphilus} is present in multiple cities in northern Florida
  \item
    \emph{P. fructiphilus} populations may be reduced by heavy pruning
  \item
    The predatory mite \emph{Amblyseius swirskii} is attracted to roses infected with Rose Rosette Disease (RRD)
  \item
    Infected roses have higher levels of defense-related terpenes, including \textgamma-Muurolene, \textbeta-Caryophyllene, and D-Limonene
  \item
    ASM-treated plants had similar more profiles to one another than to the Volatile Organic Compounds (VOCs) released from healthy or RRD-infected plants
  \item
    Systemic Acquired Resistance (SAR) via acibenzolar-S-methyl (ASM) does not appear to significantly reduce populations of \emph{P. fructiphilus}
  \item
    The integrated pest management treatments we tested were not very efficient for controlling populations of \emph{P. fructiphilus}
  \item
    Orchid Fleck Virus (OFV) is present in ornamental groundcover plants in Florida. the vector of OFV (\emph{Brevipalpus californicus}) is present as well
  \end{enumerate}
  \hypertarget{invasive-mites-are-a-problem-in-florida}{%
  \section{Invasive mites are a problem in Florida}\label{invasive-mites-are-a-problem-in-florida}}

  We encountered two invasive mite species associated with plant viruses in the last three years. This is likely an unfortunate byproduct of the plant trade.

  \hypertarget{references}{%
  \chapter*{REFERENCES}\label{references}}
  \addcontentsline{toc}{chapter}{REFERENCES}

  Placeholder

%formats and injects the bibliography here
\bibliography {bib/sample,bib/example}

%this takes the 'preface:' field from your yaml header and uses it as the biography
%removes the dependency on the .tex file
  \biography{Austin N. Fife received his B.S. degree in Biology - Zoology Emphasis at Brigham Young University - Idaho at Rexburg, a M.S. in Entomology at the University of Idaho at Moscow, and a Ph.D.~in Entomology at the University of Florida, Gainesville, where he specialized in acarology, biological control, chemical ecology, vector biology and plant-pathogen-arthropod interactions. He loves to spend time outdoors with his wife Liz and his two lovely daughters Violet and Juniper.}

\end{document}
%------------------------------------------------------------------------------%
