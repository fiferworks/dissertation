% This is the Reed College LaTeX thesis template. Most of the work
% for the document class was done by Sam Noble (SN), as well as this
% template. Later comments etc. by Ben Salzberg (BTS). Additional
% restructuring and APA support by Jess Youngberg (JY).
% Your comments and suggestions are more than welcome; please email
% them to cus@reed.edu
%
% See https://www.reed.edu/cis/help/LaTeX/index.html for help. There are a
% great bunch of help pages there, with notes on
% getting started, bibtex, etc. Go there and read it if you're not
% already familiar with LaTeX.
%
% Any line that starts with a percent symbol is a comment.
% They won't show up in the document, and are useful for notes
% to yourself and explaining commands.
% Commenting also removes a line from the document;
% very handy for troubleshooting problems. -BTS


%%%%%%%%%%%%%%
%% Preamble %%
%%%%%%%%%%%%%%
% \documentclass{<something>} must begin each LaTeX document
\documentclass{ufdissertation}[overrideChapters] %UF's 2019 Template --ANF
% Packages are extensions to the basic LaTeX functions. Whatever you
% want to typeset, there is probably a package out there for it.
% Chemistry (chemtex), screenplays, you name it.
% Check out CTAN to see: https://www.ctan.org/
%%
\usepackage{siunitx}
\usepackage{textgreek}
\usepackage[section]{placeins}
\usepackage{pdfpages}
\usepackage{calc}
\usepackage{rotating}

% Syntax highlighting #22

% So, this code uses your CSL file to decide how to format your citations
% You may need to edit your CSL if the editorial office doesn't like it
% From {rticles}
\newlength{\csllabelwidth}
\setlength{\csllabelwidth}{3em}
\newlength{\cslhangindent}
\setlength{\cslhangindent}{1.5em}
% for Pandoc 2.8 to 2.10.1
\newenvironment{cslreferences}%
  {}%
  {\par}
% For Pandoc 2.11+
% As noted by @mirh [2] is needed instead of [3] for 2.12
\newenvironment{CSLReferences}[2] % #1 hanging-ident, #2 entry spacing
 {% don't indent paragraphs
  \setlength{\parindent}{0pt}
  % turn on hanging indent if param 1 is 1
  \ifodd #1 \everypar{\setlength{\hangindent}{\cslhangindent}}\ignorespaces\fi
  % set entry spacing
  \ifnum #2 > 0
  \setlength{\parskip}{#2\baselineskip}
  \fi
 }%
 {}
\usepackage{calc} % for calculating minipage widths
\newcommand{\CSLBlock}[1]{#1\hfill\break}
\newcommand{\CSLLeftMargin}[1]{\parbox[t]{\csllabelwidth}{#1}}
\newcommand{\CSLRightInline}[1]{\parbox[t]{\linewidth - \csllabelwidth}{#1}}
\newcommand{\CSLIndent}[1]{\hspace{\cslhangindent}#1}

\providecommand{\tightlist}{%
  \setlength{\itemsep}{0pt}\setlength{\parskip}{0pt}}
% % Added by CII (Thanks, Hadley!)
% % Use ref for internal links
\renewcommand{\hyperref}[2][???]{\autoref{#1}}
\def\chapterautorefname{Chapter}
\def\sectionautorefname{Section}
\def\subsectionautorefname{Subsection}
% End of CII addition
\begin{document}

%%%%%%%%%%%%%%%%%%%%%%%%%%%%%%%%%
% TITLE PAGE                    %
%%%%%%%%%%%%%%%%%%%%%%%%%%%%%%%%%
    \begin{center}
        \thispagestyle{empty}%
        \vspace*{-0.4in}\realSingleSpace{DETECTION OF TWO MITE-PLANT-VIRUS PATHOSYSTEMS IN FLORIDA, INCLUDING FIELD STUDIES AND CHEMICAL ECOLOGY OF INDUCED PLANT DEFENSES ON \emph{Phyllocoptes fructiphilus} AND \emph{Amblyseius swirskii}}%
        \vfill%
        By \\*[\baselineskip]%
        \MakeUppercase{Austin N Fife}%
        \vfill%
        A \MakeUppercase{Dissertation} PRESENTED TO THE GRADUATE SCHOOL \\%
        OF THE UNIVERSITY OF FLORIDA IN PARTIAL FULFILLMENT \\%
        OF THE REQUIREMENTS FOR THE DEGREE OF \\%
        \MakeUppercase{Doctor of Philosophy} \\*[\baselineskip]%
        UNIVERSITY OF FLORIDA \\*[\baselineskip]%
        {2021}%
    \end{center}
    \newpage

%%%%%%%%%%%%%%%%%%%%%%%%%%%%%%%%%
% COPYRIGHT PAGE                %
%%%%%%%%%%%%%%%%%%%%%%%%%%%%%%%%%
    \newpage
      \vspace*{\fill}
        \begin{center}
            \textcopyright{} {2021} {Austin N Fife}
        \end{center}
      \vspace*{\fill}
    \newpage

\setcounter{secnumdepth}{-1}     % We don't want chapter numbers until later,
                                 % So let's kill off the table of contents depth detector until we want to start counting.
                                 

%%%%%%%%%%%%%%%%%%%%%%%%%%%%%%%%%
% DEDICATION                    %
%%%%%%%%%%%%%%%%%%%%%%%%%%%%%%%%%

    \vspace*{\fill}               % We want the dedication to be centered,
                                  % So we use \vspace*{\fill} above and below.
\begin{center}                    % We also want to center the dedication horizontally.
  \realSingleSpace
  {For Liz, Violet, Juniper and Fifes to come}
\end{center}
\vspace*{\fill}                   % Note that the * in \vspace* is necessary,
                                  % as otherwise latex will ignore it here


%%%%%%%%%%%%%%%%%%%%%%%%%%%%%%%%%
% ACKNOWLEDGMENTS               %
%%%%%%%%%%%%%%%%%%%%%%%%%%%%%%%%%

{\hypertarget{acknowledgments}{%
\chapter{ACKNOWLEDGMENTS}\label{acknowledgments}}

I would like to give special thanks for the Tallahassee Museum and their patience, cooperation, and support with collecting plant samples. I am grateful for the USDA-APHIS PPQ Beltsville laboratory for their support in the identification and confirmation OFV isolates, as well as \emph{Brevipalpus} mite identification at the USDA-ARS. I also want to thank Drs. Sam Bolton, FDACS and Aline Tassi, Univ. of Sao Paulo, Brazil for checking the mites we have sent for species validation. Furthermore, I am grateful for Dr.~Marc S. Frank's identification of the Liriopogons collected. I am especially indebted to the late Dr.~Gary Bauchan for his contributions to this study and the field of acarology, he will be greatly missed. This research was partly funded by the USDA National Institute of Food and Agriculture, Hatch project FLA-NFC-005607 and USDA-AFRI-CPPM 2017-70006-27268. Funds were also contributed by The Florida Nursery, Growers and Landscape Association's (FNGLA) Endowed Research Funds. Mention of trade names or commercial products in this publication is solely for the purpose of providing specific information and does not imply recommendation or endorsement by the USDA; USDA is an equal opportunity provider and employer.}              % Inputs the text found in the 00--acknowledgments.Rmd file


%%%%%%%%%%%%%%%%%%%%%%%%%%%%%%%%%
% TABLE OF CONTENTS             %
%%%%%%%%%%%%%%%%%%%%%%%%%%%%%%%%%
\realSingleSpace\tableofcontents  % Table of Contents comes fourth.

%%%%%%%%%%%%%%%%%%%%%%%%%%%%%%%%%
% LIST OF TABLES                %
%%%%%%%%%%%%%%%%%%%%%%%%%%%%%%%%%

\realSingleSpace\listoftables    % List of tables comes next, if you have one.
\addcontentsline{toc}{chapter}{LIST OF TABLES}


%%%%%%%%%%%%%%%%%%%%%%%%%%%%%%%%%
% LIST OF FIGURES               %
%%%%%%%%%%%%%%%%%%%%%%%%%%%%%%%%%
\realSingleSpace\listoffigures   % List of figures comes next, if you have one.
\addcontentsline{toc}{chapter}{LIST OF FIGURES}

%%%%%%%%%%%%%%%%%%%%%%%%%%%%%%%%%
% LIST OF ABBREVIATIONS         %
%%%%%%%%%%%%%%%%%%%%%%%%%%%%%%%%%

%%%%%%%%%%%%%%%%%%%%%%%%%%%%%%%%%
% ACADEMIC ABSTRACT             %
%%%%%%%%%%%%%%%%%%%%%%%%%%%%%%%%%

\newpage                         % Since the abstract needs to be a phantom chapter, we need to force a newpage.
    \phantomsection
    \addcontentsline{toc}{chapter}{ABSTRACT}
    \label{abstract}
        \begin{center}\realSingleSpace
            Abstract of Dissertation Presented to the Graduate School \\
            of the University of Florida in Partial Fulfillment of the \\
            Requirements for the Degree of Doctor of Philosophy\\[\baselineskip]
            {DETECTION OF TWO MITE-PLANT-VIRUS PATHOSYSTEMS IN FLORIDA, INCLUDING FIELD STUDIES AND CHEMICAL ECOLOGY OF INDUCED PLANT DEFENSES ON \emph{Phyllocoptes fructiphilus} AND \emph{Amblyseius swirskii}}\\[\baselineskip] % reads in your title from the YAML header
            By\\[\baselineskip]
            {Austin N Fife} \\[\baselineskip]
            {December} {2021}\\[\baselineskip]
        \end{center}
    \realSingleSpace\vspace*{-\baselineskip}
            \hfill \break
                \noindent Chair: {Xavier Martini} \\    % If we have a chair recorded, display it.
                                \noindent Cochair: {Mathews Paret} \\% If there is a Co-chair provided, then list it.
                            \noindent Major: {Entomology and Nematology} \\
   \hphantom{forcing a space here} \\
{Rose Rosette Virus (genus Emaraviridae) is the most devastating disease of roses. Rose Rosette Virus (RRV) creates witches brooms, rosetting, deforms flowers, increases prickle density, elongates shoots, reddens of plant tissues, causes dieback and ultimately plant death. RRV is spread by a microscopic eriophyid mite known as Phyllocoptes fructiphilus Keifer (Trombidiformes: Eriophyidae). Few management options are available: Current mite control is achieved by removing infected roses and frequent pesticide applications. Growers are interested in alternative and less expensive management options to combat P. fructiphilus and RRV. Aggressive pruning was tested for its ability to reduce populations of P. fructiphilus. Mites from the family Phytoseiidae are being investigated as biocontrol agents for the management of P. fructiphilus. A survey of mites on roses was conducted in northern Florida to search for P. fructiphilus, RRD and/or predatory mites. Preliminary data suggest that the phytoseiid mite Amblyseius swirskii Athias-Henriot (Mesostigmata: Phytoseiidae) orients itself towards volatiles of RRV-infected roses. This attraction may have synergistic effects for *P. fructiphilus control. A. swirskii was tested in olfactometer choice tests to identify specific volatile compounds which may cause this behavior. Low levels of Methyl Salicylate found in RRV-infected roses suggest that the virus interferes with the rose's ability to defend itself against the pathogen. A avoid this negative feedback loop is to induce Systemic Acquired Resistance (SAR) before pathogen introduction, a procedure which could increase the rose's resistance to RRV. Considering this knowledge, we collaborated with the University of Georgia in to test how SAR-induction might protect roses from P. fructiphilus and/or RRD. Acibenzolar-S-methyl (ASM) is a benzothiadiazole, a SAR-inducing chemical which works like Salicylic Acid to induce plant defenses against viruses, bacteria, and fungal infection. ASM application also has shown chitinase activity in roses, and some studies have shown that the hypersensitive response and SAR interfere with the ability of eriophyoid mites to feed or grow on induced plants. A remaining concern is that SAR-induction may harm predatory mites via direct and indirect effects of SAR-induction. We conducted several field studies from 2018-2021 in order to test the integration of predatory mites with SAR. This research will contribute a biocontrol option for the management of P. fructiphilus in southern Georgia and northern Florida. We describe the first detection of orchid fleck virus (OFV) infecting three unreported hosts: Liriope muscari, cv. `Gigantea' (Decaisne) Bailey, Ophiopogon intermedius Don and Aspidistra elatior Blume (Asparagaceae: Nolinoidaea) in Leon and Alachua Counties, FL. The orchid-infecting subgroup (Orc) of OFV infects over 50 plant species belonging to the plant families Orchidaceae, Asparagaceae (Nolinoidaea), and causes citrus leprosis disease in Citrus (Rutaceae). The only known vectors of OFV-Orc are the flat mites, Brevipalpus californicus (Banks) sensu lato (Trombidiformes: Tenuipalpidae). Florida has various plants in the landscape which Brevipalpus spp. feed on, which are susceptible to infection by OFV-Orc. Chlorotic ringspots and flecking were seen affecting Liriopogons (Liriope and Ophiopogon spp.) in Leon County, FL. Nearby A. elatior also appeared chlorotic. Local diagnostics returned negative for common plant pathogens, therefore new samples were sent to the Florida Department of Agriculture and Consumer Services (FDACS) and USDA-ARS for identification. Two orchid-infecting strains of OFV were detected via combinations of conventional RT-PCR, RT-qPCR, Sanger sequencing and High Throughput Sequencing (HTS). Amplicons shared 98\% nucleotide identity with OFV-Orc1 and OFV-Orc2 RNA2 genome sequences available in NCBI GenBank. Coinfections were detected in each county, but single strains of OFV-Orc were detected in L. muscari (Alachua, OFV-Orc2) and A. elatior (Leon, OFV-Orc1). Three potential mite vectors were identified via cryo-scanning electron microscopy (Cryo-SEM): Brevipalpus californicus (Banks) sensu lato, B. obovatus Donnadieu, and B. confusus Baker. In conclusion, OFV orchid strains are present in northern Florida, representing a risk for susceptible plants in the southeastern US.}

\addtocontents{toc}{\protect\contentsline{part}{CHAPTER}{}{}}%% Input a dummy "CHAPTER" heading to show user-content
        \setcounter{secnumdepth}{5}
        
%%%%%%%%%%%%%%%%%%%%%%%%%%%%%%%%%
% CHAPTERS                      %
%%%%%%%%%%%%%%%%%%%%%%%%%%%%%%%%%

\docBodytrue
 \doublespacing
    {\hypertarget{institution-the-university-of-florida}{%
\chapter{institution: `The University of Florida'}\label{institution-the-university-of-florida}}

Placeholder

\hypertarget{literature-review}{%
\chapter{LITERATURE REVIEW}\label{literature-review}}

Placeholder

\hypertarget{a-small-introduction-to-some-herbivorous-acari}{%
\section{A Small Introduction to Some Herbivorous Acari}\label{a-small-introduction-to-some-herbivorous-acari}}

\hypertarget{coevolved-plant-specialists-the-eriophyoidea}{%
\subsection{Coevolved plant specialists: the eriophyoidea}\label{coevolved-plant-specialists-the-eriophyoidea}}

\hypertarget{phyllocoptes-fructiphilus-the-vector-of-rose-rosette-emaravirus-the-causal-agent-of-rose-rosette-disease}{%
\subsection{\texorpdfstring{\emph{Phyllocoptes fructiphilus}: the vector of \emph{Rose rosette emaravirus}, the causal agent of Rose Rosette Disease}{Phyllocoptes fructiphilus: the vector of Rose rosette emaravirus, the causal agent of Rose Rosette Disease}}\label{phyllocoptes-fructiphilus-the-vector-of-rose-rosette-emaravirus-the-causal-agent-of-rose-rosette-disease}}

\hypertarget{integrated-pest-management-best-practices-for-modern-agriculture}{%
\section{Integrated Pest Management: Best Practices for Modern Agriculture}\label{integrated-pest-management-best-practices-for-modern-agriculture}}

\hypertarget{current-management-of-rose-rosette-disease-is-not-effective}{%
\subsection{Current management of Rose Rosette Disease is not effective}\label{current-management-of-rose-rosette-disease-is-not-effective}}

\hypertarget{small-phytoseiid-mites-could-be-an-option-for-biocontrol-if-discovered}{%
\subsection{Small phytoseiid mites could be an option for biocontrol if discovered}\label{small-phytoseiid-mites-could-be-an-option-for-biocontrol-if-discovered}}

\hypertarget{struggling-to-stay-in-roses-chemical-ecology-of-amblyseius-swirskii-in-roses}{%
\subsection{\texorpdfstring{Struggling to stay in roses: chemical ecology of \emph{Amblyseius swirskii} in roses}{Struggling to stay in roses: chemical ecology of Amblyseius swirskii in roses}}\label{struggling-to-stay-in-roses-chemical-ecology-of-amblyseius-swirskii-in-roses}}

\hypertarget{induced-plant-defenses}{%
\section{Induced Plant Defenses}\label{induced-plant-defenses}}

\hypertarget{chemeco}{%
\section{Can Systemic Acquired Resistance Reduce Mite Herbivory?}\label{chemeco}}

\hypertarget{a-second-plant-mite-pathosystem-brevipalpus-californicus-and-orchid-fleck-dichorhavirus}{%
\section{\texorpdfstring{A Second Plant-Mite-Pathosystem: \emph{Brevipalpus californicus} and \emph{Orchid fleck dichorhavirus}}{A Second Plant-Mite-Pathosystem: Brevipalpus californicus and Orchid fleck dichorhavirus}}\label{a-second-plant-mite-pathosystem-brevipalpus-californicus-and-orchid-fleck-dichorhavirus}}

\hypertarget{survey-and-phenology-of-natural-populations-of-the-invasive-mite-phyllocoptes-fructiphilus-in-northern-florida}{%
\chapter{\texorpdfstring{SURVEY AND PHENOLOGY OF NATURAL POPULATIONS OF THE INVASIVE MITE \emph{Phyllocoptes fructiphilus} IN NORTHERN FLORIDA}{SURVEY AND PHENOLOGY OF NATURAL POPULATIONS OF THE INVASIVE MITE Phyllocoptes fructiphilus IN NORTHERN FLORIDA}}\label{survey-and-phenology-of-natural-populations-of-the-invasive-mite-phyllocoptes-fructiphilus-in-northern-florida}}

Placeholder

\hypertarget{introduction}{%
\section{Introduction}\label{introduction}}

\hypertarget{surveying-for-phyllocoptes-fructiphilus-rose-rosette-disease-and-predatory-mites-in-northern-florida}{%
\section{\texorpdfstring{Surveying for \emph{Phyllocoptes fructiphilus}, Rose Rosette Disease and Predatory Mites in Northern Florida}{Surveying for Phyllocoptes fructiphilus, Rose Rosette Disease and Predatory Mites in Northern Florida}}\label{surveying-for-phyllocoptes-fructiphilus-rose-rosette-disease-and-predatory-mites-in-northern-florida}}

\hypertarget{materials-methods}{%
\section{Materials \& Methods}\label{materials-methods}}

\hypertarget{results}{%
\section{Results}\label{results}}

\hypertarget{discussion}{%
\section{Discussion}\label{discussion}}

\hypertarget{changes-in-headspace-volatiles-for-roses-infected-with-rose-rosette-disease}{%
\chapter{CHANGES IN HEADSPACE VOLATILES FOR ROSES INFECTED WITH ROSE ROSETTE DISEASE}\label{changes-in-headspace-volatiles-for-roses-infected-with-rose-rosette-disease}}

Placeholder

\hypertarget{introduction-1}{%
\section{Introduction}\label{introduction-1}}

\hypertarget{plant-defenses-and-volatiles-why-are-amblyseius-swirskii-attracted-infected-roses}{%
\subsection{\texorpdfstring{Plant defenses and volatiles: why are \emph{Amblyseius swirskii} attracted infected roses?}{Plant defenses and volatiles: why are Amblyseius swirskii attracted infected roses?}}\label{plant-defenses-and-volatiles-why-are-amblyseius-swirskii-attracted-infected-roses}}

\hypertarget{materials-methods-1}{%
\section{Materials \& Methods}\label{materials-methods-1}}

\hypertarget{collection-of-headspace-volatiles-from-roses}{%
\subsection{Collection of headspace volatiles from roses}\label{collection-of-headspace-volatiles-from-roses}}

\hypertarget{volatile-collection-trap-methodology}{%
\subsubsection{Volatile collection trap methodology}\label{volatile-collection-trap-methodology}}

\hypertarget{solid-phase-micro-extraction-spme-methodology}{%
\subsubsection{Solid phase micro extraction (SPME) methodology}\label{solid-phase-micro-extraction-spme-methodology}}

\hypertarget{analysis-of-headspace-data}{%
\subsubsection{Analysis of headspace data}\label{analysis-of-headspace-data}}

\hypertarget{which-volatiles-are-most-attractive-two-arm-olfactometer-assays}{%
\subsection{Which volatiles are most attractive?: two-arm olfactometer assays}\label{which-volatiles-are-most-attractive-two-arm-olfactometer-assays}}

\hypertarget{results-1}{%
\section{Results}\label{results-1}}

\hypertarget{volatiles-differ-between-rrd-infected-uninfected-and-induced-roses}{%
\subsection{Volatiles differ between RRD-infected, uninfected and induced roses}\label{volatiles-differ-between-rrd-infected-uninfected-and-induced-roses}}

\hypertarget{amblyseius-swirskii-were-not-affected-by-the-vocs-tested}{%
\subsection{\texorpdfstring{\emph{Amblyseius swirskii} were not affected by the VOCs tested}{Amblyseius swirskii were not affected by the VOCs tested}}\label{amblyseius-swirskii-were-not-affected-by-the-vocs-tested}}

\hypertarget{discussion-1}{%
\section{Discussion}\label{discussion-1}}

\hypertarget{management-of-phyllocoptes-fructiphiulus-with-systemic-acquired-resistance}{%
\chapter{\texorpdfstring{MANAGEMENT OF \emph{Phyllocoptes fructiphiulus} WITH SYSTEMIC ACQUIRED RESISTANCE}{MANAGEMENT OF Phyllocoptes fructiphiulus WITH SYSTEMIC ACQUIRED RESISTANCE}}\label{management-of-phyllocoptes-fructiphiulus-with-systemic-acquired-resistance}}

Placeholder

\hypertarget{introduction-phyllocoptes-fructiphiulus---an-increasingly-large-problem}{%
\section{\texorpdfstring{Introduction: \emph{Phyllocoptes fructiphiulus} - An Increasingly Large Problem}{Introduction: Phyllocoptes fructiphiulus - An Increasingly Large Problem}}\label{introduction-phyllocoptes-fructiphiulus---an-increasingly-large-problem}}

\hypertarget{integrating-pest-management-what-are-the-effects-of-systemic-acquired-resistance-on-amblyseius-swirskii-and-pphyllocoptes-fructiphilus}{%
\subsection{\texorpdfstring{Integrating pest management: what are the effects of systemic acquired resistance on \emph{Amblyseius swirskii} and \emph{PPhyllocoptes fructiphilus}?}{Integrating pest management: what are the effects of systemic acquired resistance on Amblyseius swirskii and PPhyllocoptes fructiphilus?}}\label{integrating-pest-management-what-are-the-effects-of-systemic-acquired-resistance-on-amblyseius-swirskii-and-pphyllocoptes-fructiphilus}}

\hypertarget{phenology-of-populations-of-phyllocoptes-fructiphilus-in-northern-florida}{%
\subsection{\texorpdfstring{Phenology of populations of \emph{Phyllocoptes fructiphilus} in northern Florida}{Phenology of populations of Phyllocoptes fructiphilus in northern Florida}}\label{phenology-of-populations-of-phyllocoptes-fructiphilus-in-northern-florida}}

\hypertarget{materials-methods-2}{%
\section{Materials \& Methods}\label{materials-methods-2}}

\hypertarget{inducing-systemic-acquired-resistance-with-acibenzolar-s-methyl-to-reduce-populations-of-phyllocoptes-fructiphilus}{%
\subsection{\texorpdfstring{Inducing systemic acquired resistance with acibenzolar-S-methyl to reduce populations of \emph{Phyllocoptes fructiphilus}}{Inducing systemic acquired resistance with acibenzolar-S-methyl to reduce populations of Phyllocoptes fructiphilus}}\label{inducing-systemic-acquired-resistance-with-acibenzolar-s-methyl-to-reduce-populations-of-phyllocoptes-fructiphilus}}

\hypertarget{integrating-management-options-to-control-phyllocoptes-fructiphilus}{%
\subsection{\texorpdfstring{Integrating management options to control \emph{Phyllocoptes fructiphilus}}{Integrating management options to control Phyllocoptes fructiphilus}}\label{integrating-management-options-to-control-phyllocoptes-fructiphilus}}

\hypertarget{phenology-field-study}{%
\subsection{Phenology field study}\label{phenology-field-study}}

\hypertarget{ipm-field-trials-tallahassee-2021}{%
\subsection{IPM field trials, Tallahassee 2021}\label{ipm-field-trials-tallahassee-2021}}

\hypertarget{analysis-of-field-trial-data}{%
\subsection{Analysis of field trial data}\label{analysis-of-field-trial-data}}

\hypertarget{results-2}{%
\section{Results}\label{results-2}}

\hypertarget{asm-trials}{%
\subsection{ASM trials}\label{asm-trials}}

\hypertarget{phenology}{%
\subsection{Phenology}\label{phenology}}

\hypertarget{ipm-trials}{%
\subsection{IPM trials}\label{ipm-trials}}

\hypertarget{discussion-2}{%
\section{Discussion}\label{discussion-2}}

\hypertarget{brevipalpus-transmitted-orchid-fleck-dichorhavirus-infecting-three-new-ornamental-hosts-in-florida}{%
\chapter{\texorpdfstring{\emph{Brevipalpus}-TRANSMITTED \emph{Orchid fleck dichorhavirus} INFECTING THREE NEW ORNAMENTAL HOSTS IN FLORIDA}{Brevipalpus-TRANSMITTED Orchid fleck dichorhavirus INFECTING THREE NEW ORNAMENTAL HOSTS IN FLORIDA}}\label{brevipalpus-transmitted-orchid-fleck-dichorhavirus-infecting-three-new-ornamental-hosts-in-florida}}

Placeholder

\hypertarget{virus-detection}{%
\section{Virus Detection}\label{virus-detection}}

\hypertarget{a-comment-on-the-status-of-brevipalpus-in-florida}{%
\section{\texorpdfstring{A Comment on the Status of \emph{Brevipalpus} in Florida}{A Comment on the Status of Brevipalpus in Florida}}\label{a-comment-on-the-status-of-brevipalpus-in-florida}}

\hypertarget{conclusions-invasive-mites-are-an-increasingly-large-problem-for-florida}{%
\chapter{CONCLUSIONS: INVASIVE MITES ARE AN INCREASINGLY LARGE PROBLEM FOR FLORIDA}\label{conclusions-invasive-mites-are-an-increasingly-large-problem-for-florida}}

\hypertarget{invasive-species-are-a-problem-in-florida}{%
\section{Invasive Species are a Problem in Florida}\label{invasive-species-are-a-problem-in-florida}}

The aim of our research was to address the emerging problems caused by the recent invasions of mites and the plant viruses they vector to northern Florida. Florida is a hotspot for invasive species in the mainland US, due to a combination of its unique geography, climate, rapid development and a multitude of invasion pathways (\protect\hyperlink{ref-Simberloff1997}{Simberloff et al. 1997}, \protect\hyperlink{ref-Williams2007}{Williams et al. 2007}, \protect\hyperlink{ref-Card2018}{Card et al. 2018}). Invasive species are often harmful to ecosystem services and modify ecosystems to the detriment of native species (\protect\hyperlink{ref-Gordon1998}{Gordon 1998}). Efforts to manage invasive species costs the state of Florida \textasciitilde\$45M each year (\protect\hyperlink{ref-Hiatt2019}{Hiatt et al. 2019}). Many Florida programs have been developed to raise public awareness about the threats of invasive species, as well as encouraging the reporting of invasive species by the public (\protect\hyperlink{ref-Wallace2016}{Wallace et al. 2016}). \emph{Phyllocoptes fructiphilus}, Rose Rosette Disease, and Orchid Fleck Virus all represent yet another chapter in the story of Florida invasion ecology. These organisms have a high possibility of becoming established invasive species in Florida if preventative actions are not taken. Our research was designed to address two of the main issues of controlling \emph{P. fructiphilus}: early detection and integrated pest management of the mite vector.

\hypertarget{main-findings}{%
\section{Main Findings}\label{main-findings}}

The principal findings of our experiments are:
\begin{enumerate}
\def\labelenumi{\arabic{enumi}.}
\tightlist
\item
  \emph{Phyllocoptes fructiphilus} is present in multiple cities in northern Florida
\item
  The predatory mite \emph{Amblyseius swirskii} is attracted to roses infected with Rose Rosette Disease (RRD)
\item
  Infected roses have higher levels of defense-related terpenes, including \textgamma-Muurolene, \textbeta-Caryophyllene, and D-Limonene, some of which may be attractive to \emph{A. swirskii}
\item
  ASM-treated plants had more similar profiles to one another than to the Volatile Organic Compounds (VOCs) released from healthy or RRD-infected plants
\item
  Systemic Acquired Resistance (SAR) via acibenzolar-S-methyl (ASM) does not appear to significantly reduce populations of \emph{P. fructiphilus}
\item
  \emph{P. fructiphilus} populations may be reduced by spirotetramat applications, or possibly heavy pruning
\item
  Orchid Fleck Virus (OFV) is present in ornamental groundcover plants in Florida. the vector of OFV (\emph{Brevipalpus californicus}, \emph{sensu lato}) is present as well
\end{enumerate}
We encountered two invasive mite species associated with plant viruses in the last three years. This is likely an unfortunate byproduct of the plant trade and the (\protect\hyperlink{ref-Chapman2017}{Chapman et al. 2017}).

\docBodyfalse
\setcounter{secnumdepth}{-1}

\hypertarget{references}{%
\chapter*{REFERENCES}\label{references}}
\addcontentsline{toc}{chapter}{REFERENCES}

\realSingleSpace

\hypertarget{refs}{}
\begin{CSLReferences}{1}{1}
\leavevmode\vadjust pre{\hypertarget{ref-Baker1987}{}}%
\textbf{Baker, E. W., and D. M. Tuttle}. \textbf{1987}. The false spider mites of {Mexico} ({Tenuipalpidae}: {Acari}). (technical report No. 1706). The {United States} Department of Agriculture - Agricultural Research Service.

\leavevmode\vadjust pre{\hypertarget{ref-Bates2015}{}}%
\textbf{Bates, D., M. Mächler, B. Bolker, and S. Walker}. \textbf{2015}. Fitting linear mixed-effects models using {lme4}. Journal of Statistical Software. 67: 1--48, DOI:\href{https://doi.org/10.18637/jss.v067.i01}{10.18637/jss.v067.i01}.

\leavevmode\vadjust pre{\hypertarget{ref-Beard2015}{}}%
\textbf{Beard, J. J., R. Ochoa, W. E. Braswell, and G. R. Bauchan}. \textbf{2015}. {\emph{Brevipalpus phoenicis}} {(Geijskes)} species complex ({Acari}: {Tenuipalpidae}) \textemdash a closer look. Zootaxa. 3944: 1, DOI:\href{https://doi.org/10.11646/zootaxa.3944.1.1}{10.11646/zootaxa.3944.1.1}.

\leavevmode\vadjust pre{\hypertarget{ref-Bischl2016}{}}%
\textbf{Bischl, B., M. Lang, L. Kotthoff, J. Schiffner, J. Richter, E. Studerus, G. Casalicchio, and Z. M. Jones}. \textbf{2016}. \href{https://jmlr.org/papers/v17/15-066.html}{{mlr}: Machine learning in {R}}. Journal of Machine Learning Research. 17: 1--5.

\leavevmode\vadjust pre{\hypertarget{ref-Broussard2007}{}}%
\textbf{Broussard, M. C.} \textbf{2007}. A horticultural study of {\emph{Liriope}} and {\emph{Ophiopogon}}: Nomenclature, morphology, and culture (PhD thesis). Louisiana State University, Department of Horticulture.

\leavevmode\vadjust pre{\hypertarget{ref-vanBuuren2011}{}}%
\textbf{Buuren, S. van, and K. Groothuis-Oudshoorn}. \textbf{2011}. \href{https://www.jstatsoft.org/v45/i03/}{{mice}: {Multivariate} {Imputation} by {Chained} {Equations} in {R}}. Journal of Statistical Software. 45: 1--67.

\leavevmode\vadjust pre{\hypertarget{ref-Card2018}{}}%
\textbf{Card, D. C., B. W. Perry, R. H. Adams, D. R. Schield, A. S. Young, A. L. Andrew, T. Jezkova, G. I. M. Pasquesi, N. R. Hales, M. R. Walsh, M. R. Rochford, F. J. Mazzotti, K. M. Hart, M. E. Hunter, and T. A. Castoe}. \textbf{2018}. Novel ecological and climatic conditions drive rapid adaptation in invasive {Florida} {Burmese} pythons. 27: 4744--4757, DOI:\href{https://doi.org/10.1111/mec.14885}{10.1111/mec.14885}.

\leavevmode\vadjust pre{\hypertarget{ref-Chang2019}{}}%
\textbf{Chang, W.} \textbf{2019}. \href{https://CRAN.R-project.org/package=webshot}{{webshot:} Take screenshots of web pages}.

\leavevmode\vadjust pre{\hypertarget{ref-Chapman2017}{}}%
\textbf{Chapman, D., B. V. Purse, H. E. Roy, and J. M. Bullock}. \textbf{2017}. Global trade networks determine the distribution of invasive non-native species. Global Ecology and Biogeography. 26: 907--917, DOI:\href{https://doi.org/10.1111/geb.12599}{10.1111/geb.12599}.

\leavevmode\vadjust pre{\hypertarget{ref-Eddelbuettel2018}{}}%
\textbf{Eddelbuettel, D., and J. J. Balamuta}. \textbf{2018}. {Extending \emph{R} with \emph{C++}: A Brief Introduction to \emph{Rcpp}}. The American Statistician. 72: 28--36, DOI:\href{https://doi.org/10.1080/00031305.2017.1375990}{10.1080/00031305.2017.1375990}.

\leavevmode\vadjust pre{\hypertarget{ref-Fantz2008b}{}}%
\textbf{Fantz, P. R.} \textbf{2008}. Species of {\emph{Liriope}} cultivated in the southeastern {United States}. {HortTechnology}. 18: 343--348, DOI:\href{https://doi.org/10.21273/horttech.18.3.343}{10.21273/horttech.18.3.343}.

\leavevmode\vadjust pre{\hypertarget{ref-Fantz2009}{}}%
\textbf{Fantz, P. R.} \textbf{2009}. Names and species of {\emph{Ophiopogon}} cultivated in the southeastern {United States}. {HortTechnology}. 19: 385--394, DOI:\href{https://doi.org/10.21273/hortsci.19.2.385}{10.21273/hortsci.19.2.385}.

\leavevmode\vadjust pre{\hypertarget{ref-Fantz2015}{}}%
\textbf{Fantz, P. R., D. Carey, T. Avent, and J. Lattier}. \textbf{2015}. Inventory, descriptions, and keys to segregation and identification of liriopogons cultivated in the southeastern {United States}. {HortScience}. 50: 957--993, DOI:\href{https://doi.org/10.21273/hortsci.50.7.957}{10.21273/hortsci.50.7.957}.

\leavevmode\vadjust pre{\hypertarget{ref-Feinerer2008}{}}%
\textbf{Feinerer, I., K. Hornik, and D. Meyer}. \textbf{2008}. \href{https://www.jstatsoft.org/v25/i05/}{Text mining infrastructure in {R}}. Journal of Statistical Software. 25: 1--54.

\leavevmode\vadjust pre{\hypertarget{ref-Fox2019}{}}%
\textbf{Fox, J., and S. Weisberg}. \textbf{2019}. \href{https://socialsciences.mcmaster.ca/jfox/Books/Companion/}{An {R} companion to applied regression}, Third. ed. Sage, Thousand Oaks {CA}.

\leavevmode\vadjust pre{\hypertarget{ref-Gordon1998}{}}%
\textbf{Gordon, D. R.} \textbf{1998}. Effects of invasive, non-indigenous plant species on ecosystem processes: Lessons from {Florida}. Ecological Applications. 8: 975--989, DOI:\href{https://doi.org/10.1890/1051-0761(1998)008\%5B0975:eoinip\%5D2.0.co;2}{10.1890/1051-0761(1998)008{[}0975:eoinip{]}2.0.co;2}.

\leavevmode\vadjust pre{\hypertarget{ref-Grolemund2011}{}}%
\textbf{Grolemund, G., and H. Wickham}. \textbf{2011}. \href{https://www.jstatsoft.org/v40/i03/}{Dates and times made easy with {lubridate}}. Journal of Statistical Software. 40: 1--25.

\leavevmode\vadjust pre{\hypertarget{ref-Henry2021}{}}%
\textbf{Henry, L., and H. Wickham}. \textbf{2021}. \href{https://CRAN.R-project.org/package=tidyselect}{{tidyselect:} Select from a set of strings}.

\leavevmode\vadjust pre{\hypertarget{ref-Hiatt2019}{}}%
\textbf{Hiatt, D., K. Serbesoff-King, D. Lieurance, D. R. Gordon, and S. L. Flory}. \textbf{2019}. Allocation of invasive plant management expenditures for conservation: Lessons from {Florida}, {USA}. 1, DOI:\href{https://doi.org/10.1111/csp2.51}{10.1111/csp2.51}.

\leavevmode\vadjust pre{\hypertarget{ref-Hothorn2008}{}}%
\textbf{Hothorn, T., F. Bretz, and P. Westfall}. \textbf{2008}. Simultaneous inference in general parametric models. Biometrical Journal. 50: 346--363.

\leavevmode\vadjust pre{\hypertarget{ref-Jackman2020}{}}%
\textbf{Jackman, S.} \textbf{2020}. \href{https://github.com/atahk/pscl/}{{pscl}: Classes and methods for {R} developed in the political science computational laboratory}. United States Studies Centre, University of Sydney, Sydney, New South Wales, Australia.

\leavevmode\vadjust pre{\hypertarget{ref-Lenth2021}{}}%
\textbf{Lenth, R. V.} \textbf{2021}. \href{https://CRAN.R-project.org/package=emmeans}{{emmeans}: Estimated marginal means, aka least-squares means}.

\leavevmode\vadjust pre{\hypertarget{ref-Masiero2020}{}}%
\textbf{Masiero, E., D. Banik, J. Abson, P. Greene, A. Slater, and T. Sgamma}. \textbf{2020}. Molecular verification of the {UK} national collection of cultivated {\emph{Liriope}} and {\emph{Ophiopogon}} plants. Plants. 9: 558, DOI:\href{https://doi.org/10.3390/plants9050558}{10.3390/plants9050558}.

\leavevmode\vadjust pre{\hypertarget{ref-Mueller2021}{}}%
\textbf{Müller, K., and H. Wickham}. \textbf{2021}. \href{https://CRAN.R-project.org/package=tibble}{{tibble}: Simple data frames}.

\leavevmode\vadjust pre{\hypertarget{ref-Neuwirth2014}{}}%
\textbf{Neuwirth, E.} \textbf{2014}. \href{https://CRAN.R-project.org/package=RColorBrewer}{{RColorBrewer}: Colorbrewer palettes}.

\leavevmode\vadjust pre{\hypertarget{ref-Ottolinger2019}{}}%
\textbf{Ottolinger, P.} \textbf{2019}. \href{https://CRAN.R-project.org/package=bib2df}{{bib2df}: Parse a {BibTeX} file to a data frame}.

\leavevmode\vadjust pre{\hypertarget{ref-Schloerke2021}{}}%
\textbf{Schloerke, B., D. Cook, J. Larmarange, F. Briatte, M. Marbach, E. Thoen, A. Elberg, and J. Crowley}. \textbf{2021}. \href{https://CRAN.R-project.org/package=GGally}{{GGally}: Extension to 'ggplot2'}.

\leavevmode\vadjust pre{\hypertarget{ref-Simberloff1997}{}}%
\textbf{(Strangers in paradise ) }. \textbf{1997}. Strangers in paradise. Island Press, Washington, D.C.

\leavevmode\vadjust pre{\hypertarget{ref-Tennekes2018}{}}%
\textbf{Tennekes, M.} \textbf{2018}. {tmap}: Thematic maps in {R}. Journal of Statistical Software. 84: 1--39, DOI:\href{https://doi.org/10.18637/jss.v084.i06}{10.18637/jss.v084.i06}.

\leavevmode\vadjust pre{\hypertarget{ref-Tierney2021}{}}%
\textbf{Tierney, N., D. Cook, M. McBain, and C. Fay}. \textbf{2021}. \href{https://CRAN.R-project.org/package=naniar}{Naniar: Data structures, summaries, and visualisations for missing data}.

\leavevmode\vadjust pre{\hypertarget{ref-Wallace2016}{}}%
\textbf{Wallace, R. D., C. T. Bargeron, D. J. Moorhead, and J. H. LaForest}. \textbf{2016}. {IveGot}1: Reporting and tracking invasive species in {Florida}. 15: 51--62, DOI:\href{https://doi.org/10.1656/058.015.sp805}{10.1656/058.015.sp805}.

\leavevmode\vadjust pre{\hypertarget{ref-Wang2014}{}}%
\textbf{Wang, G.-Y., Y. Meng, J.-L. Huang, and Y.-P. Yang}. \textbf{2014}. Molecular phylogeny of {\emph{Ophiopogon}} {({Asparagaceae})} inferred from nuclear and plastid {DNA} sequences. Systematic Botany. 39: 776--784, DOI:\href{https://doi.org/10.1600/036364414x682201}{10.1600/036364414x682201}.

\leavevmode\vadjust pre{\hypertarget{ref-Wickham2019b}{}}%
\textbf{Wickham, H.} \textbf{2019}. \href{https://CRAN.R-project.org/package=stringr}{{stringr}: Simple, consistent wrappers for common string operations}.

\leavevmode\vadjust pre{\hypertarget{ref-Wickham2021b}{}}%
\textbf{Wickham, H., J. Hester, and W. Chang}. \textbf{2021}. \href{https://CRAN.R-project.org/package=devtools}{{devtools}: Tools to make developing {R} packages easier}.

\leavevmode\vadjust pre{\hypertarget{ref-Williams2007}{}}%
\textbf{Williams, D. A., E. Muchugu, W. A. Overholt, and J. P. Cuda}. \textbf{2007}. Colonization patterns of the invasive {Brazilian} peppertree, {\emph{Schinus terebinthifolius}}, in {Florida}. 98: 284--293, DOI:\href{https://doi.org/10.1038/sj.hdy.6800936}{10.1038/sj.hdy.6800936}.

\leavevmode\vadjust pre{\hypertarget{ref-Zeileis2008}{}}%
\textbf{Zeileis, A., C. Kleiber, and S. Jackman}. \textbf{2008}. \href{http://www.jstatsoft.org/v27/i08/}{Regression models for count data in {R}}. Journal of Statistical Software. 27.

\end{CSLReferences}
\doublespacing

\hypertarget{biographical-sketch}{%
\chapter*{BIOGRAPHICAL SKETCH}\label{biographical-sketch}}
\addcontentsline{toc}{chapter}{BIOGRAPHICAL SKETCH}

\realSingleSpace\vspace*{\baselineskip}

Austin N Fife received his B.S. degree in biology - zoology emphasis at Brigham Young University - Idaho at Rexburg, a M.S. in entomology at the University of Idaho at Moscow, and a Ph.D.~in entomology and nematology at the University of Florida, Gainesville, where he specialized in acarology, biological control, chemical ecology, vector biology and plant-pathogen-arthropod interactions. He loves to spend time outdoors with his wife Liz and his two lovely daughters Violet and Juniper.

\includepdf[pages=1,pagecommand=\chapter{APPENDIX}, offset=0 -3cm]{pubs/pfruct-report.pdf}
\includepdf[pages={2-last}, pagecommand={\thispagestyle{plain}}, fitpaper = true]{pubs/pfruct-report.pdf}}                       % This imports the body chapters included in the _bookdown.yml file,
% you can add as many chapters as you need

%%%%%%%%%%%%%%%%%%%%%%%%%%%%%%%%%
% APPENDICES                    %
%%%%%%%%%%%%%%%%%%%%%%%%%%%%%%%%%

% If you have multiple appendices, add them as new chapters in the
% prelims/99--appendix file and edit the chapter headings to say 'APPENDIX A'
% 'APPENDIX B' and so on


%%%%%%%%%%%%%%%%%%%%%%%%%%%%%%%%%
% LIST OF REFERENCES            %
%%%%%%%%%%%%%%%%%%%%%%%%%%%%%%%%%

% provided by file 99-references.Rmd

%%%%%%%%%%%%%%%%%%%%%%%%%%%%%%%%%
% BIOGRAPHICAL SKETCH
%%%%%%%%%%%%%%%%%%%%%%%%%%%%%%%%%

% provided by file prelims/99--biography.Rmd


\end{document}
